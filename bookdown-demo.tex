% Options for packages loaded elsewhere
\PassOptionsToPackage{unicode}{hyperref}
\PassOptionsToPackage{hyphens}{url}
%
\documentclass[
]{book}
\usepackage{amsmath,amssymb}
\usepackage{lmodern}
\usepackage{iftex}
\ifPDFTeX
  \usepackage[T1]{fontenc}
  \usepackage[utf8]{inputenc}
  \usepackage{textcomp} % provide euro and other symbols
\else % if luatex or xetex
  \usepackage{unicode-math}
  \defaultfontfeatures{Scale=MatchLowercase}
  \defaultfontfeatures[\rmfamily]{Ligatures=TeX,Scale=1}
\fi
% Use upquote if available, for straight quotes in verbatim environments
\IfFileExists{upquote.sty}{\usepackage{upquote}}{}
\IfFileExists{microtype.sty}{% use microtype if available
  \usepackage[]{microtype}
  \UseMicrotypeSet[protrusion]{basicmath} % disable protrusion for tt fonts
}{}
\makeatletter
\@ifundefined{KOMAClassName}{% if non-KOMA class
  \IfFileExists{parskip.sty}{%
    \usepackage{parskip}
  }{% else
    \setlength{\parindent}{0pt}
    \setlength{\parskip}{6pt plus 2pt minus 1pt}}
}{% if KOMA class
  \KOMAoptions{parskip=half}}
\makeatother
\usepackage{xcolor}
\IfFileExists{xurl.sty}{\usepackage{xurl}}{} % add URL line breaks if available
\IfFileExists{bookmark.sty}{\usepackage{bookmark}}{\usepackage{hyperref}}
\hypersetup{
  pdftitle={Statistical Techniques for Biological and Environmental Sciences},
  pdfauthor={Brad Duthie},
  hidelinks,
  pdfcreator={LaTeX via pandoc}}
\urlstyle{same} % disable monospaced font for URLs
\usepackage{longtable,booktabs,array}
\usepackage{calc} % for calculating minipage widths
% Correct order of tables after \paragraph or \subparagraph
\usepackage{etoolbox}
\makeatletter
\patchcmd\longtable{\par}{\if@noskipsec\mbox{}\fi\par}{}{}
\makeatother
% Allow footnotes in longtable head/foot
\IfFileExists{footnotehyper.sty}{\usepackage{footnotehyper}}{\usepackage{footnote}}
\makesavenoteenv{longtable}
\usepackage{graphicx}
\makeatletter
\def\maxwidth{\ifdim\Gin@nat@width>\linewidth\linewidth\else\Gin@nat@width\fi}
\def\maxheight{\ifdim\Gin@nat@height>\textheight\textheight\else\Gin@nat@height\fi}
\makeatother
% Scale images if necessary, so that they will not overflow the page
% margins by default, and it is still possible to overwrite the defaults
% using explicit options in \includegraphics[width, height, ...]{}
\setkeys{Gin}{width=\maxwidth,height=\maxheight,keepaspectratio}
% Set default figure placement to htbp
\makeatletter
\def\fps@figure{htbp}
\makeatother
\setlength{\emergencystretch}{3em} % prevent overfull lines
\providecommand{\tightlist}{%
  \setlength{\itemsep}{0pt}\setlength{\parskip}{0pt}}
\setcounter{secnumdepth}{5}
\usepackage{booktabs}
\usepackage{amsthm}
\makeatletter
\def\thm@space@setup{%
  \thm@preskip=8pt plus 2pt minus 4pt
  \thm@postskip=\thm@preskip
}
\makeatother
\ifLuaTeX
  \usepackage{selnolig}  % disable illegal ligatures
\fi
\usepackage[]{natbib}
\bibliographystyle{apalike}

\title{Statistical Techniques for Biological and Environmental Sciences}
\author{Brad Duthie}
\date{2022-07-12}

\begin{document}
\maketitle

{
\setcounter{tocdepth}{1}
\tableofcontents
}
\hypertarget{preface}{%
\chapter*{Preface}\label{preface}}
\addcontentsline{toc}{chapter}{Preface}

Welcome to the module. This workbook will be used throughout the semester and contain all of the information that you need for the statistical techniques (SCIU4T4) module.

\hypertarget{what-is-statistics}{%
\section{What is statistics?}\label{what-is-statistics}}

An explanation of the material, and what will be taught.

\hypertarget{why-this-module-is-important}{%
\section{Why this module is important}\label{why-this-module-is-important}}

Some discussion of module importance

\hypertarget{teaching-overview}{%
\section{Teaching overview}\label{teaching-overview}}

Here is how you will be taught, with online lectures, reading assignments, and face-to-face practicals.

\hypertarget{assessment-overview}{%
\section{Assessment overview}\label{assessment-overview}}

You will have one formative test and two summatitive tests. You will also have one mock exam and one exam exam.

\hypertarget{test-1f}{%
\subsection{Test 1F}\label{test-1f}}

Information about Test 1F

\hypertarget{test-1s}{%
\subsection{Test 1S}\label{test-1s}}

Information about Test 1S

\hypertarget{test-2s}{%
\subsection{Test 2S}\label{test-2s}}

Information about Test 1S

\hypertarget{mock-exam}{%
\subsection{Mock Exam}\label{mock-exam}}

Information about the mock exam

\hypertarget{exam}{%
\subsection{Exam}\label{exam}}

Information about the exam

\hypertarget{jamovi-statistical-software}{%
\section{Jamovi statistical software}\label{jamovi-statistical-software}}

Introduction to \href{https://www.jamovi.org/}{Jamovi}, and why we are using it instead of other software.

\hypertarget{textbooks}{%
\section{Textbooks}\label{textbooks}}

Introduction to the primary textbook \href{https://www.learnstatswithjamovi.com/}{Learning statistics with jamovi}, and a mention of other sources.

\hypertarget{canvas}{%
\section{Canvas}\label{canvas}}

How we will use Canvas, and how this book relates to it (Learning and Teaching content, where lectures, assessments, and discussions can be found).

\hypertarget{timetable}{%
\section{Timetable}\label{timetable}}

\hypertarget{part-background-mathematics-and-data-organisation}{%
\part{Background mathematics and data organisation}\label{part-background-mathematics-and-data-organisation}}

In week 1, we will focus on a refresher of some necessary background mathematics for this module. We will then turn to the topic of how to organise data sets. We will then practice organising datasets and saving them in a usable format.

Week: 1
Dates:
Suggested Readings: Textbook intro, Hadley's paper
Assessments: Practice quiz
Practical: Quick summary of topics covered

\hypertarget{background_mathematics}{%
\chapter{Background mathematics}\label{background_mathematics}}

Some of this will be review, but it is important. Suggested reading for this (some mathematics text).

\hypertarget{numbers-and-operations}{%
\section{Numbers and operations}\label{numbers-and-operations}}

A very broad reminder of mathematics, which you will need for this module

\hypertarget{order-of-operations}{%
\section{Order of operations}\label{order-of-operations}}

This is easy to forget

\hypertarget{data_organisation}{%
\chapter{Data organisation}\label{data_organisation}}

It is important to organise data properly so that statistical analysis can be done. Here I explain the tidy approach to data. Suggested reading Hadley Wickam's paper.

\hypertarget{tidy-data}{%
\section{Tidy data}\label{tidy-data}}

\hypertarget{data-files}{%
\section{Data files}\label{data-files}}

\hypertarget{practical-preparing-real-datasets}{%
\chapter{\texorpdfstring{\emph{Practical}: Preparing real datasets}{Practical: Preparing real datasets}}\label{practical-preparing-real-datasets}}

In this practical, we will use a spreadsheet to organise datastes.

\hypertarget{libreoffice-calc}{%
\section{LibreOffice Calc}\label{libreoffice-calc}}

LibreOffice Calc is a free and open source spreadsheet program. The instructions for this section will be identical to the more popular commercial Microsoft Excel.

\hypertarget{exercise-organising-data-1}{%
\section{Exercise Organising data 1}\label{exercise-organising-data-1}}

Walks through Exercise 1.3.2

\hypertarget{exercise-organising-data-2}{%
\section{Exercise Organising data 2}\label{exercise-organising-data-2}}

Walks through Exercise 1.3.3

\hypertarget{exercise-organising-data-3}{%
\section{Exercise Organising data 3}\label{exercise-organising-data-3}}

Walks through Exercise 1.3.4, saving all of these as CSV files

\hypertarget{summary-of-exercises-and-why-they-are-useful.}{%
\section{Summary of exercises and why they are useful.}\label{summary-of-exercises-and-why-they-are-useful.}}

\hypertarget{part-statistical-concepts}{%
\part{Statistical concepts}\label{part-statistical-concepts}}

Overview of what this week will include.

Week: 2
Dates:
Suggested Readings: Textbook intro to Jamovi
Assessments: Practice quiz

\hypertarget{recap-of-some-statistical-concepts}{%
\chapter{Recap of some statistical concepts}\label{recap-of-some-statistical-concepts}}

Some introduction

\hypertarget{why-study-statistics}{%
\chapter{Why study statistics?}\label{why-study-statistics}}

General discussion

\hypertarget{populations-and-samples}{%
\chapter{Populations and samples}\label{populations-and-samples}}

Explanation of the mode

\hypertarget{types-of-variables}{%
\chapter{Types of variables}\label{types-of-variables}}

Categorical, ordinal, continuous, etc.

\hypertarget{units-precision-and-accuracy}{%
\chapter{Units, precision, and accuracy}\label{units-precision-and-accuracy}}

Resistance in statistics

\hypertarget{uncertainty-propogation}{%
\chapter{Uncertainty propogation}\label{uncertainty-propogation}}

Some simple equations

\hypertarget{practical.-introduction-to-jamovi}{%
\chapter{\texorpdfstring{\emph{Practical}. Introduction to Jamovi}{Practical. Introduction to Jamovi}}\label{practical.-introduction-to-jamovi}}

Some introductory text to Jamovi. Particular attention, and maybe an example, on different data types and how to find them in Jamovi.

\hypertarget{exercise-for-summary-statistics}{%
\section{Exercise for summary statistics}\label{exercise-for-summary-statistics}}

\hypertarget{exercise-to-compute-variable}{%
\section{Exercise to compute variable}\label{exercise-to-compute-variable}}

\hypertarget{exercise-on-transforming-variables}{%
\section{Exercise on transforming variables}\label{exercise-on-transforming-variables}}

\hypertarget{part-summary-statistics}{%
\part{Summary statistics}\label{part-summary-statistics}}

Overview of what this week will include.

Week: 3
Dates:
Suggested Readings: Textbook intro to Jamovi
Assessments: Practice quiz

\hypertarget{decimal-places-and-significant-figures}{%
\chapter{Decimal places and significant figures}\label{decimal-places-and-significant-figures}}

It is important to know how to write a number you have calculated to the appropriate number of digits, typically either defined as a number of decimal places or as a number of `significant figures'. This is especially important in any module (for example, this one) where you are required to put numeric answers into a test or exam, as the computer will only recognise the answer as being correct if it is expressed in the way stated in the question. If you are unfamiliar with how to work out the right number of decimal places or significant figures, then these guides should be useful, although there are many other web-sources that would explain the concepts too.

\hypertarget{the-mean}{%
\chapter{The mean}\label{the-mean}}

Explanation of the mean

\hypertarget{the-mode}{%
\chapter{The mode}\label{the-mode}}

Explanation of the mode

\hypertarget{the-median-and-quantiles}{%
\chapter{The median and quantiles}\label{the-median-and-quantiles}}

Notes on the median and quantiles

\hypertarget{mean-mode-median-and-resistance}{%
\chapter{Mean, mode, median, and resistance}\label{mean-mode-median-and-resistance}}

Resistance in statistics

\hypertarget{plots}{%
\chapter{Plots}\label{plots}}

Graphics are critical for visualising data, which is always important

\hypertarget{general-principles}{%
\section{General principles}\label{general-principles}}

Some points about plots

\hypertarget{histograms}{%
\section{Histograms}\label{histograms}}

Histograms are special, and introduce the concept of a distribution.

\hypertarget{box-whisker-plots}{%
\section{Box-whisker plots}\label{box-whisker-plots}}

\hypertarget{practical.-real-data-with-jamovi}{%
\chapter{\texorpdfstring{\emph{Practical}. Real data with Jamovi}{Practical. Real data with Jamovi}}\label{practical.-real-data-with-jamovi}}

Using some real datasets in Jamovi

\hypertarget{some-biological-example}{%
\section{Some biological example}\label{some-biological-example}}

Remember to first put it in a tidy format. Get summary statistics here too.

\hypertarget{some-environmental-example}{%
\section{Some environmental example}\label{some-environmental-example}}

Transform and compute a new variable, plotting in Jamovi.

\hypertarget{summary-of-exercises-and-why-they-are-useful.-1}{%
\section{Summary of exercises and why they are useful.}\label{summary-of-exercises-and-why-they-are-useful.-1}}

Useful for scientific publications, reading, and dissertation work.

\hypertarget{part-probability-models-and-the-central-limit-theorem}{%
\part{Probability models and the Central Limit Theorem}\label{part-probability-models-and-the-central-limit-theorem}}

General overview of what will be the focus of this week.

Week: 4
Dates:
Suggested Readings: Textbook intro to probability
Assessments: Practice quiz
Practical:

\hypertarget{introduction-to-probability-models}{%
\chapter{Introduction to probability models}\label{introduction-to-probability-models}}

Some background

\hypertarget{a-practical-example}{%
\section{A practical example}\label{a-practical-example}}

How to think about probability

\hypertarget{probability-distributions}{%
\section{Probability distributions}\label{probability-distributions}}

Some more useful examples

\hypertarget{binomial-distribution}{%
\subsection{Binomial distribution}\label{binomial-distribution}}

Explanation, fairly straightforward

\hypertarget{poisson-distribution}{%
\subsection{Poisson distribution}\label{poisson-distribution}}

Another example

\hypertarget{normal-distribution}{%
\subsection{Normal distribution}\label{normal-distribution}}

Why this is so important

\hypertarget{the-central-limit-theorem-clt}{%
\chapter{The Central Limit Theorem (CLT)}\label{the-central-limit-theorem-clt}}

General overview

\hypertarget{examples-of-the-clt-in-action}{%
\section{Examples of the CLT in action}\label{examples-of-the-clt-in-action}}

\hypertarget{the-standard-normal-distribution}{%
\section{The standard normal distribution}\label{the-standard-normal-distribution}}

\hypertarget{what-are-z-scores}{%
\section{What are z-scores?}\label{what-are-z-scores}}

\hypertarget{practical.-probability-and-simulation}{%
\chapter{\texorpdfstring{\emph{Practical}. Probability and simulation}{Practical. Probability and simulation}}\label{practical.-probability-and-simulation}}

Some of these examples will be similar to what will be on the assessment

\hypertarget{calculating-probability-exercise-1}{%
\section{Calculating probability exercise 1}\label{calculating-probability-exercise-1}}

Example exercise 1 with some simple probability calculations

\hypertarget{calculating-probability-exercise-2}{%
\section{Calculating probability exercise 2}\label{calculating-probability-exercise-2}}

Example exercise 2 with some simple probability calculations

\hypertarget{calculating-probability-from-normal-distribution}{%
\section{Calculating probability from normal distribution}\label{calculating-probability-from-normal-distribution}}

Example exercise for getting a the probability of some value sampled above, below, or between some threshold in Jamovi.

\hypertarget{normal-distribution-and-sample-size}{%
\section{Normal distribution and sample size}\label{normal-distribution-and-sample-size}}

Showing how we get closer to the normal distribution as sample size increases in Jamovi.

\hypertarget{simulating-the-central-limit-theorem}{%
\section{Simulating the central limit theorem}\label{simulating-the-central-limit-theorem}}

Doing the example from a uniform distribution in Jamovi.

\hypertarget{part-statistical-inference}{%
\part{Statistical inference}\label{part-statistical-inference}}

General overview of what will be the focus of this week.

Week: 5
Dates:
Suggested Readings: Textbook intro to probability
Assessments: Practice quiz
Practical:

\hypertarget{sample-statistics-and-population-parameters}{%
\chapter{Sample statistics and population parameters}\label{sample-statistics-and-population-parameters}}

An explanation of this

\hypertarget{standard-normal-distribution}{%
\chapter{Standard Normal Distribution}\label{standard-normal-distribution}}

What this means, and why it is important.

\hypertarget{confidence-intervals}{%
\chapter{Confidence intervals}\label{confidence-intervals}}

How these are calculated, and how to interpret them

\hypertarget{the-t-interval}{%
\chapter{The t-interval}\label{the-t-interval}}

What this is and how it relates to the normal distribution, and why it is important.

\hypertarget{practical.-z--and-t--intervals}{%
\chapter{\texorpdfstring{\emph{Practical}. z- and t- intervals}{Practical. z- and t- intervals}}\label{practical.-z--and-t--intervals}}

\hypertarget{example-constructing-confidence-intervals}{%
\section{Example constructing confidence intervals}\label{example-constructing-confidence-intervals}}

\hypertarget{confidence-interval-for-different-levels-t--and-z-}{%
\section{Confidence interval for different levels (t- and z-)}\label{confidence-interval-for-different-levels-t--and-z-}}

\hypertarget{proportion-confidence-intervals}{%
\section{Proportion confidence intervals}\label{proportion-confidence-intervals}}

\hypertarget{another-confidence-interval-example}{%
\section{Another confidence interval example?}\label{another-confidence-interval-example}}

\hypertarget{part-review-of-parts-i-v}{%
\part{Review of parts I-V}\label{part-review-of-parts-i-v}}

This is a special chapter for week 6, which is a reading week, and it will function as a very brief pause for review. It will also ensure that the numbers of chapters will correspond to weeks.

\hypertarget{part-hypothesis-testing}{%
\part{Hypothesis testing}\label{part-hypothesis-testing}}

General overview of what will be the focus of this week.

Week: 7
Dates:
Suggested Readings: Textbook
Assessments: Practice quiz
Practical:

\hypertarget{what-is-hypothesis-testing}{%
\chapter{What is hypothesis testing?}\label{what-is-hypothesis-testing}}

An explanation of this, and that we are starting to get into some of the more interesting bits of inferential statistics.

\hypertarget{making-and-using-hypotheses-and-types-of-tests}{%
\chapter{Making and using hypotheses and types of tests}\label{making-and-using-hypotheses-and-types-of-tests}}

What this means, and why it is important.

\hypertarget{an-example-of-hypothesis-testing}{%
\chapter{An example of hypothesis testing}\label{an-example-of-hypothesis-testing}}

Errors

\hypertarget{hypothesis-testing-and-confidence-intervals}{%
\chapter{Hypothesis testing and confidence intervals}\label{hypothesis-testing-and-confidence-intervals}}

Relationship between these two.

\hypertarget{student-t-distribution-and-one-sample-t-test}{%
\chapter{Student t-distribution and one sample t-test}\label{student-t-distribution-and-one-sample-t-test}}

What this is and how to do it in Jamovi.

\hypertarget{another-example-of-a-one-sample-t-test}{%
\chapter{Another example of a one sample t-test}\label{another-example-of-a-one-sample-t-test}}

From the lectures

\hypertarget{independent-t-test}{%
\chapter{Independent t-test}\label{independent-t-test}}

What this is and how to use it in Jamovi.

\hypertarget{paired-sample-t-test}{%
\chapter{Paired sample t-test}\label{paired-sample-t-test}}

Another explanation, example, and how to do it in Jamovi.

\hypertarget{violations-of-assumptions}{%
\chapter{Violations of assumptions}\label{violations-of-assumptions}}

What to do in this case

\hypertarget{non-parametric-tests-and-what-these-are.}{%
\chapter{Non-parametric tests, and what these are.}\label{non-parametric-tests-and-what-these-are.}}

Explanation of how to do them in Jamovi.

\hypertarget{practical.-hypothesis-testing-and-t-tests}{%
\chapter{\texorpdfstring{\emph{Practical}. Hypothesis testing and t-tests}{Practical. Hypothesis testing and t-tests}}\label{practical.-hypothesis-testing-and-t-tests}}

\hypertarget{exercise-on-a-simple-one-sample-t-test}{%
\section{Exercise on a simple one sample t-test}\label{exercise-on-a-simple-one-sample-t-test}}

\hypertarget{exercise-on-an-independent-sample-t-test}{%
\section{Exercise on an independent sample t-test}\label{exercise-on-an-independent-sample-t-test}}

\hypertarget{exercise-involving-multiple-comparisons}{%
\section{Exercise involving multiple comparisons}\label{exercise-involving-multiple-comparisons}}

\hypertarget{exercise-with-non-parametric}{%
\section{Exercise with non-parametric}\label{exercise-with-non-parametric}}

\hypertarget{another-exercise-with-non-parametric}{%
\section{Another exercise with non-parametric}\label{another-exercise-with-non-parametric}}

\hypertarget{part-analysis-of-variance-anova}{%
\part{Analysis of Variance (ANOVA)}\label{part-analysis-of-variance-anova}}

General overview of what will be the focus of this week.

Week: 8
Dates:
Suggested Readings: Textbook
Assessments: Practice quiz
Practical:

\hypertarget{what-is-anova}{%
\chapter{What is ANOVA?}\label{what-is-anova}}

General explanation

\hypertarget{one-way-anova}{%
\chapter{One-way ANOVA}\label{one-way-anova}}

Explain what this is.

\hypertarget{two-way-anova}{%
\chapter{Two-way ANOVA}\label{two-way-anova}}

More explanation

\hypertarget{kruskall-wallis-h-test}{%
\chapter{Kruskall-Wallis H test}\label{kruskall-wallis-h-test}}

Non-parametric explanation

\hypertarget{practical.-anova-and-associated-tests}{%
\chapter{\texorpdfstring{\emph{Practical}. ANOVA and associated tests}{Practical. ANOVA and associated tests}}\label{practical.-anova-and-associated-tests}}

\hypertarget{anova-exercise-1}{%
\section{ANOVA Exercise 1}\label{anova-exercise-1}}

\hypertarget{anova-exercise-2}{%
\section{ANOVA Exercise 2}\label{anova-exercise-2}}

\hypertarget{anova-exercise-3}{%
\section{ANOVA Exercise 3}\label{anova-exercise-3}}

\hypertarget{anova-exercise-4}{%
\section{ANOVA Exercise 4}\label{anova-exercise-4}}

\hypertarget{part-counts-and-correlation}{%
\part{Counts and Correlation}\label{part-counts-and-correlation}}

General overview of what will be the focus of this week.

Week: 9
Dates:
Suggested Readings: Textbook
Assessments: Practice quiz
Practical:

\hypertarget{frequency-and-count-data}{%
\chapter{Frequency and count data}\label{frequency-and-count-data}}

General explanation

\hypertarget{chi-squared-goodness-of-fit}{%
\chapter{Chi-squared goodness of fit}\label{chi-squared-goodness-of-fit}}

Explain what this is.

\hypertarget{chi-squared-test-of-association}{%
\chapter{Chi-squared test of association}\label{chi-squared-test-of-association}}

More explanation

\hypertarget{correlation-key-concepts}{%
\chapter{Correlation key concepts}\label{correlation-key-concepts}}

\hypertarget{correlation-mathematics}{%
\chapter{Correlation mathematics}\label{correlation-mathematics}}

\hypertarget{correlation-hypothesis-testing}{%
\chapter{Correlation hypothesis testing}\label{correlation-hypothesis-testing}}

\hypertarget{practical.-analysis-of-count-data-correlation-and-regression}{%
\chapter{\texorpdfstring{\emph{Practical}. Analysis of count data, correlation, and regression}{Practical. Analysis of count data, correlation, and regression}}\label{practical.-analysis-of-count-data-correlation-and-regression}}

\hypertarget{chi-square-exercise-1}{%
\section{Chi-Square Exercise 1}\label{chi-square-exercise-1}}

\hypertarget{chi-square-association-exercise-2}{%
\section{Chi-Square association Exercise 2}\label{chi-square-association-exercise-2}}

\hypertarget{correlation-exercise-3}{%
\section{Correlation Exercise 3}\label{correlation-exercise-3}}

\hypertarget{correlation-exercise-4}{%
\section{Correlation Exercise 4}\label{correlation-exercise-4}}

\hypertarget{part-linear-regression}{%
\part{Linear Regression}\label{part-linear-regression}}

General overview of what will be the focus of this week.

Week: 10
Dates:
Suggested Readings: Textbook
Assessments: Practice quiz
Practical:

\hypertarget{regression-key-concepts}{%
\chapter{Regression key concepts}\label{regression-key-concepts}}

\hypertarget{regression-validity}{%
\chapter{Regression validity}\label{regression-validity}}

\hypertarget{introduction-to-multiple-regression}{%
\chapter{Introduction to multiple regression}\label{introduction-to-multiple-regression}}

General explanation

\hypertarget{model-selection-maybe-remove-this}{%
\chapter{Model selection (maybe remove this?)}\label{model-selection-maybe-remove-this}}

Seriously consider moving the regression into this week. and ease the amount of material in previous weeks.

\hypertarget{practical.-using-regression}{%
\chapter{\texorpdfstring{\emph{Practical}. Using regression}{Practical. Using regression}}\label{practical.-using-regression}}

\hypertarget{regression-exercise-1}{%
\section{Regression Exercise 1}\label{regression-exercise-1}}

\hypertarget{regression-exercise-2}{%
\section{Regression Exercise 2}\label{regression-exercise-2}}

\hypertarget{regression-exercise-3}{%
\section{Regression Exercise 3}\label{regression-exercise-3}}

\hypertarget{regression-exercise-4}{%
\section{Regression Exercise 4}\label{regression-exercise-4}}

\hypertarget{part-randomisation-approaches}{%
\part{Randomisation approaches}\label{part-randomisation-approaches}}

The aim of this lecture is to introduce the randomisation approach to statistical hypothesis testing. We will first introduce the general idea of what randomisation is and how it relates to the hypothesis testing that we have been doing since week five. We will then consider an instructive example in which a randomisation approach is used in place of a traditional t-test to test whether or not the mean values of two different groups are identical. We will then compare the assumptions underlying randomisation and how they differ slightly from the assumptions of traditional hypothesis testing. We will then look at how randomisation can be used to build confidence intervals and test hypotheses that would difficult to test with other approaches. In learning about randomisation approaches, we will also review some key concepts from earlier in the module. The aim is not to understand all of the nuances of randomisation, but to understand, conceptually, what is going on in the methods described below.

Week: 11
Dates:
Suggested Readings: Textbook
Assessments: Practice quiz
Practical: R starts creeping in now?

\hypertarget{introduction-to-randomisation}{%
\chapter{Introduction to randomisation}\label{introduction-to-randomisation}}

General explanation

\hypertarget{assumptions-of-randomisation}{%
\chapter{Assumptions of randomisation}\label{assumptions-of-randomisation}}

How these differ

\hypertarget{bootstrapping}{%
\chapter{Bootstrapping}\label{bootstrapping}}

What this is and why we use it.

\hypertarget{monte-carlo}{%
\chapter{Monte Carlo}\label{monte-carlo}}

\hypertarget{practical.-using-r}{%
\chapter{\texorpdfstring{\emph{Practical}. Using R}{Practical. Using R}}\label{practical.-using-r}}

\hypertarget{r-exercise-1}{%
\section{R Exercise 1}\label{r-exercise-1}}

\hypertarget{r-exercise-2}{%
\section{R Exercise 2}\label{r-exercise-2}}

\hypertarget{r-exercise-3}{%
\section{R Exercise 3}\label{r-exercise-3}}

\hypertarget{part-statistical-reporting}{%
\part{Statistical Reporting}\label{part-statistical-reporting}}

Week: 12
Dates:
Suggested Readings: Textbook
Assessments: Practice quiz
Practical: R starts creeping in now?

\hypertarget{reporting-statistics}{%
\chapter{Reporting statistics}\label{reporting-statistics}}

General explanation

\hypertarget{more-introduction-to-r}{%
\chapter{More introduction to R}\label{more-introduction-to-r}}

How these differ

\hypertarget{more-getting-started-with-r}{%
\chapter{More getting started with R}\label{more-getting-started-with-r}}

Just more to do.

\hypertarget{practical.-using-r-1}{%
\chapter{\texorpdfstring{\emph{Practical}. Using R}{Practical. Using R}}\label{practical.-using-r-1}}

\hypertarget{r-exercise-1-1}{%
\section{R Exercise 1}\label{r-exercise-1-1}}

\hypertarget{r-exercise-2-1}{%
\section{R Exercise 2}\label{r-exercise-2-1}}

\hypertarget{r-exercise-3-1}{%
\section{R Exercise 3}\label{r-exercise-3-1}}

\hypertarget{part-review-of-parts-vii-xii}{%
\part{Review of parts (VII-XII)}\label{part-review-of-parts-vii-xii}}

This chapter will be specifically to prepare for exam.

\hypertarget{appendix-appendix}{%
\appendix}


\hypertarget{statistical-tables}{%
\chapter{Statistical tables}\label{statistical-tables}}

  \bibliography{book.bib,packages.bib}

\end{document}
